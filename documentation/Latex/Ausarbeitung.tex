\documentclass[11pt]{article}
    % General document formatting
    \usepackage[margin=0.7in]{geometry}
    \usepackage[parfill]{parskip}
    \usepackage[utf8]{inputenc}
    \usepackage{listings}
    \lstset{
  		basicstyle=\ttfamily,
  		columns=fullflexible,
  		frame=single,
  		breaklines=true,
	}
    
    % Related to math
    \usepackage{amsmath,amssymb,amsfonts,amsthm}
    
    % Document configuration
    \linespread{1.3}

\begin{document}

Hoffmann \& Sorn, 19.04.2019, Blockseminar Qualitätssicherung: Circle CI

\section {Was ist Circle CI?}

Circle CI ist eine Plattform für kontinuierliche Integration, die sowohl als Software as a Service
als auch zur Installation auf eigener Hardware angeboten wird. Circle CI unterstützt dabei eine
große Auswahl an gängigen Sprachen und Umgebungen mit ausführlich dokumentierten Beispielprojekten,
an denen sich der Einsatz des Tools auch für CI-Anfänger einfach erlernen lässt.

\subsection {Was ist Kontinuierliche Integration?}

Die Idee hinter Kontinuierliche Integration ist, die einzelnen Komponenten einer Software während
derer Entwicklung immer wieder zu einer Einheit zusammenzufügen. Üblicherweise beinhaltet dieser
Vorgang nicht nur das Bauen des gesamten Codes, sondern auch das Ausführen von Tests, die
sicherstellen sollen, dass sich durch die neu hinzugekommenen Änderungen keine Fehler in den
Code eingeschlichen haben. Das Ziel dieses kontinuierlichen Bauens und Testens ist, die Qualität
der entwickelten Software zu erhöhen, indem man Fehler frühzeitig bemerkt und verbessert.

Um ein eigenes Projekt kontinuierlich integrieren zu können, müssen ein paar Grundvoraussetzungen
erfüllt sein. Der erste wichtige Schritt ist eine gemeinsame Codebasis, an der ein oder mehrere
Entwickler arbeiten können. Eine Möglichkeit dafür sind Versionsverwaltung wie GIT oder SVN,
in die alle Entwickler ihre Änderungen am Code einpflegen können. Eine weitere Voraussetzung
ist die Möglichkeit, den Code automatisch bauen zu lassen. Für die meisten Sprachen gibt es
hierfür Build-Tools, in denen die einzelnen Schritte des Übersetzen beschrieben werden können,
um diese später automatisiert auszuführen. Beispiele wären hierfür Make oder CMake für C und
C++ Projekte oder Maven für Java Projekte. Diese übernehmen nicht nur das übersetzen des eigenen
Codes, sondern auch das Anbinden bereits bestehender Projekte, die als Abhängigkeiten definiert
werden. Ziel hierbei ist es am Ende, die gesamte Software im besten Fall durch einen Befehl bauen
zu können, ohne dass weitere Schritte von Hand notwendig sind.
Eine weitere wichtige Voraussetzung ist es, Tests zu schreiben, mit denen sichergestellt werden
kann, dass die Änderungen richtig funktionieren und nicht in bestehendem Code zu Fehlern führt.
Hier können Tools wie SonarQube verwendet werden, die automatisch prüfen, welcher Anteil des
Codes durch Tests bereits abgedeckt sind. Eine Erweiterung dieses Konzeptes wäre die Kontinuierliche
Auslieferung. Diese beschreibt die automatisierte Auslieferung der Software, wenn diese erfolgreich
integriert wurde. So könnte beispielsweise eine Webanwendung immer aktualisiert werden, sobald neue
Features fertig entwickelt und getestet sind.

Auch wenn Kontinuierliche Integration gerade am Anfang mit zusätzlichem Aufwand verbunden ist, gibt es
auf lange Sicht doch viele Vorteile, die den Einsatz in den meisten etwas umfangreicheren
Softwareprojekten sinnvoll machen.
Tauchen bei der Entwicklung Fehler auf, so können diese schnell und zeitnah behoben werden.
Zudem ermöglicht es kontinuierliche Integration, immer eine sehr aktuelle Version der Software für
Demonstrationen für beispielsweise Vertriebszwecke zur Verfügung zu haben.
Eine Gefahr bei zu sehr langen Integrationszyklen ist es, dass die gemeinsame Codebasis sich zu
stark in der Zeit ändert, in der ein neues, großes Feature entwickelt wird. Dies führt dazu, dass
das einbinden des neuen Features zu einer langwierigen und komplizierten Aufgabe wird. Werden die
Integrationszyklen hingegen klein gehalten, so wird der Gesamtaufwand einer Integration deutlich
verringert.

\section{Einführung in die Verwendung von Circle CI}
Eine wichtige Frage bei der Einführung neuer Tools in ein Projekt ist immer, welche Vorteile und
Nachteile sich aus der Verwendung ergeben. Bei Circle CI sehen wir uns dabei verstärkt die SaaS
Variante an, da diese uns gerade am Anfang administrativen Aufwand und die Notwendigkeit von
extra Hardware für die Installation erspart. Für die Integration des eigenen Projekts in Circle CI
macht es im Grunde genommen keinen Unterschied, ob Code für das Projekt bereits geschrieben wurde,
oder ob man mit einem neuen Projekt beginnt, in dem es weder Code noch Tests gibt. Im Code selbst
sind keine Änderungen notwendig. Das Projekt benötigt lediglich eine Datei mit dem Namen
\textit{config.yml} in einem Ordner \textit{.circleci}, den man in das Root Verzeichnis des Projektes
ablegt, das in Circle CI gebaut werden soll. Für Circle CI stellt diese Datei eine Anleitung mit
einzelnen Arbeitsschritten bereit. Diese Datei in Verbindung mit der Autorisierung des Repositories
genügen, um das Projekt von Circle CI bauen und testen zu können. Auf die genaue Konfiguration, die
in der YAML Datei hinterlegt wird, gehen wir später etwas genauer anhand eines praktische Beispiels
ein.

Über die Weboberfläche von Circle CI kann dann ausgewählt werden, welche der eigenen Projekte
kontinuierlich gebaut werden sollen. Wird CircleCI beispielsweise in Verbindung mit GitHub betrieben,
indem sich der Nutzer über seine GitHub Credentials authentifiziert, werden alle verfügbaren Projekte
bereits in Dieser aufgelistet. Die Projekte, die gebaut werden sollen, müssen dann nur noch ausgewählt
werden. Der Aufwand, eigene Projekte mit Circle CI kontinuierlich zu integrieren, ist damit minimal. 

\subsection{Funktionsweise von Circle CI}
Circle CI testet Projekte in jeweils eigenen Containern oder Virtuellen Maschinen, die auf die Bedürfnisse
der Projekte angepasst sind. Neben Linux basierten Containern und VMs können allerdings auch macOS
basierte "Single Use" VMs für das Testen von iOS und macOS Anwendungen verwendet werden.
Damit stellt Circle CI sicher, dass mehrere Jobs sehr einfach parallel laufen können, ohne dass sich
diese gegenseitig beeinflussen. So wird sichergestellt, dass jeder Build in der selben, sauberen
Umgebung ausgeführt wird.

\section {Vergleich zu anderen Tools und Anbieter}
\subsection {Vergleich zum Open Source Automatisierungsserver Jenkins}
Jenkins ist ein quelloffener Java basierter Automatisierungsserver, der sehr oft für kontinuierliche
Integration und Auslieferung zum Einsatz kommt.
Jenkins bietet selbst somit keine Software as a Service Lösung an, die private Anwender sofort nutzen
können.
Im Vergleich zu Circle CI bedeutet dies für den Nutzer allerdings gerade am Anfang einen erhöhten
Aufwand für Administration und höhere Kosten, da eigene 
Hardware bereitgestellt werden muss, auf der Jenkins installiert wird. Ein weiterer Unterschied zu Circle CI
ist zudem, dass die Funktionalität von Jenkins stark von Plugins abhängig ist. Funktionalitäten wie das
Auschecken von Git Repositories werden bei Jenkins durch Plugins realisiert.
Auf der einen Seite erhöht dies die Flexibilität, auf der anderen Seite erhöht dies wiederum den
administrativen Aufwand erhöht. Jenkins führt die zu automatisierenden Aufgaben in einem gewöhnlichen
Verzeichnis im Dateisystem des Hosts aus. Dabei gibt es von Haus aus keine Abschottung des Testcodes zum
Host Betriebssystem, auf dem die Jenkins Instanz läuft. Werden nun mehrere Aufgaben gleichzeitig über
Multithreading realisiert, so sind dieses nicht von einander strikt getrennt und können sich gegenseitig
beeinflussen. Möchte man die Tests in getrennten Bereichen ausführen, so muss dies vom Nutzer selbst
umgesetzt werden. Wie bereits beschrieben setzt CircleCI im Gegensatz dazu von Haus aus auf von einander
abgeschottete Container oder Virtuelle Maschinen. Eine parallelisierte Ausführung von Aufgaben, die
gegenseitig strickt von einander getrennt sind, ist damit sehr einfach umzusetzen.

Abschließend kommt es auf den Umfang des Projekts und die Kapazität von Entwicklern und Administratoren
an, welche Lösung besser geeignet ist. Sind die Projekte kleiner, gibt es wenig Kapazität für
administrative Aufgaben und soll das Projekt so schnell wie möglich in kontinuierlich integriert
werden, lohnt sich Circle CI oder eine andere Software as a Service Lösung. Wachsen die Projekte
jedoch im Umfang und Entwicklungszeit, so lohnt sich der administrative Aufwand und das Anschaffen
neuer Hardwarekapazitäten für Jenkins möglicherweise, da man hier gerade bei länger Entwicklungs-,
Test- und Übersetzungsdauer auf lange Sicht die monatlichen Kosten von Circle CI einspart.
Jenkins ist mit seinen vielen Plugins allerdings auch flexibler, während Erweiterungen bei Circle CI
selbst entwickelt und integriert werden müssen.

\subsection {Vergleich zu SaaS CI Lösung Travis}
Travis CI ist im Gegensatz zu Jenkins genau wie Circle CI eine Cloud basierte CI Lösung. Die meisten
der Vor- und Nachteile von Circle CI treffen somit auch auf Travis CI zu. Es existieren allerdings
trotzdem einige kleinere Unterschiede zwischen den beiden CI Tools. Der erste Unterschied sind die
Preise der beiden Tools. Travis CI ist lediglich für Open Source Projekte kostenlos verfügbar, während
Circle CI auch für Geschäftskonten oder private Code Repositories bis zu bestimmten Grenzen kostenlos
bleibt. Gerade für kleine Projekte mit geringer Build Zeit ist Circle CI daher besser geeignet, wenn der
Code nicht öffentlich zugänglich ist. Für Open Source Projekte hingegen bietet Travis CI jedoch mehr
Freiheiten. Im Gegensatz zu Circle CI unterstützt Travis CI Builds und Tests in einer Windows Umgebung.
Circle CI plant diese allerdings nachzuliefern. Circle CI bietet dafür wiederum eine geringere Queueing
Zeit für Builds, sowie bessere Kontrolle über Hardware Ressourcen. Ein weiterer Vorteil von Circle CI
ist, die breitere Skalierbarkeit für größere Software Projekte. Dabei werden die Pakete mit wachsender
Leistung allerdings auch stark im Preis. Travis CI bietet auf der anderen Seite einen breiter gefächerten
Support für mehr Sprachen, sowie Build Matrizen, die es erlauben, Tests in verschiedenen Sprach- und
Umgebungsversionen auszuführen.

Eine einfache Wahl zwischen den beiden Plattformen ist hier nur unter Berücksichtigung vieler Details
machbar. Eine einfache Empfehlung ist aufgrund ihrer Ähnlichkeit nur schwer zu geben.

\section{Beispielprojekte}
Nach den grundlegenden Konzepten von Circle CI bietet es sich besonders an, ein eigenes kleines
Beispielprojekt in Circle CI zu integrieren. Hierbei gehen wir in den nächsten zwei Unterabschnitte
auf zwei sehr überschaubare Beispiele ein.

\subsection{Python Projekt}
Das erste kleine Projekt ist ein in Python geschriebener Quelltext, dass auf Basis eines Master-Passwortes,
eines Nutzernamen und eines Dienstes ein Passwort generieren soll, das einfach mit Hilfe des Master
Passwortes wiederherstellbar sein soll. Die Implementierung ist hierbei zweitrangig und stellt
keinerlei Ansprüche auf Sicherheit.
Die Generierung des Passwortes lässt sich allerdings sehr gut mit einfachen Unit Tests überprüfen.
Daneben legen wir eine weitere Klasse an, die auf dem PasswordManager aufbauen soll.
Auch hier legen wir einen weiteren Test an, der die Funktionalität der Klasse prüfen soll.
Fügen wir nun Änderungen in der PasswortManager Klasse hinzu, so müssen wir in diesem Fall darauf achten,
dass wir dabei auch Klassen anpassen, die vom PasswordManager abhängig sind. Eine Gefahr wäre nun, dass
genau die nicht gemacht wird und der Fehler unentdeckt bleibt, wenn wir nach den Änderungen des
PasswordManager lediglich lokal dessen Tests ausführen.
Da wir allerdings nicht immer wissen welche Klassen und Methoden inhaltlich zusammenhängen ist es auch 
schwer, alle "betroffenen" Tests auszuführen, womit uns lediglich übrig bleibt, alle Tests auszuführen
im Projekt gibt.
Hier kommt Circle CI ins Spiel, da hiermit alle Tests ausgeführt werden können, ohne den Entwickler und
dessen Rechner für eine längere Zeit zu blockieren.

Bis hier haben wir keine Anstrengung unternommen unseren Code kontinuierlich integrieren zu können. Wie
in der Anleitung bereits erwähnt, benötigen wir auch keinerlei Änderungen am Code, sondern lediglich eine
Anleitung, was Circle CI mit dem Projekt tun soll. Hierfür legen wir in unserem Projekt einen Ordner
\textit{.circleci} mit einer Datei \textit{config.yml} an. In der ausführlichen Dokumentation von Circle CI
findet sich eine Vielzahl von Felden, die den Ablauf der Tests beeinflussen. Auf den genauen Inhalt der Konfigurationsdatei gehen wir etwas später genauer ein.

\subsection{Latex Dokumenten Generation}

Nach Integration des Python Projektes stellte sich die Frage, wie welche Aufgaben Circle CI noch erledigen kann, auch wenn diese nicht offiziell unterstützt werden. Daher entschieden wir uns zu versuchen, auch das Bauen dieses Dokuments als Job in Circle CI zu integrieren, obwohl keinerlei Unterstützung für das Übersetzen von Latex Dateien geboten wird. Für die Übersetzung benötigen wir lediglich eine Umgebung, in der eine passende Latex Distribution bereitgestellt ist, die das Dokument mit einem einfachen \textit{pdflatex} Befehl übersetzt werden kann. Circle CI bietet allerdings keine Maschine oder Docker Container, die für diese Aufgabe geeignet war. Daher entschieden wir uns für die Einbindung des Docker Containers \textit{koppor/texlive}, der nicht von Circle CI selbst stammt, aber problemlos funktionieren soll.

\subsection{Konfiguration der Circle CI Pipeline}
Für unser eigenes Projekt genügen jedoch nur einige wenige. Die Konfigurationsdatei beginnt mit der Angabe der Version von Circle CI, die verwendet werden soll. Wir entscheiden uns für die Aktuellste Version 2.1. Das nächste Feld listet die Jobs auf, die Circle CI ausführen soll. Beide werden später bei der Angabe ihrer Ausführungsreihenfolge über ihre Namen referenziert. In unserem Fall beschreiben wir die beiden Jobs über die Angabe der Umgebung die verwendet werden soll, als auch über die Befehle, die ausgeführt werden sollen. In beiden Fällen entscheiden wir uns für die Verwendung eines Docker Containers, der alle wichtigen Abhängigkeiten wie eine Python Installation oder eine Latex Distribution für das Übersetzen unseres Dokument mitbringt. Alternativ könnten wir hier auch Virtuelle Maschinen auf Linux oder macOS Basis verwenden, wenn wir diese für unseren Job benötigen würden.
Im nächsten Schritt definieren wir noch die Arbeitsschritte die für unseren Job notwendig sind. Diese werden in einer Liste beschrieben, die Circle CI dann sequentiell abarbeitet. In beiden Fällen beginnen wir dann mit dem auschecken des Projektes aus der Versionsverwaltung, welche wir mit dem Keyword \textit{checkout} anstoßen können. Im Anschluass folgen die nächsten Arbeitsschritte, die durch das Keyword 

\section{Anhang}

\lstinputlisting[label={lst:pwdmanager},caption={Implementation des the Password Managers},language=Python]{../../example_project/passwordmanager/Passwordmanager.py}
\lstinputlisting[label={lst:pwdmtest},caption={Tests für die Password Manager Klasse},language=Python]{../../example_project/test/PasswordManagerTest.py}
\lstinputlisting[label={lst:pwdmuser},caption={Klasse die von Änderungen in der Password Manager Klasse betroffen sein kann},language=Python]{../../example_project/passwordmanager_user/PasswordManagerUser.py}
\lstinputlisting[label={lst:pwdmutest},caption={Tests für die Klasse PasswordManagerUser},language=Python]{../../example_project/test/PasswordManagerUserTest.py}
\lstinputlisting[label={lst:cciconfig},caption={YML Konfiguration für Circle CI},language=Python]{../../.circleci/config.yml}
\lstinputlisting[label={lst:latexsh},caption={Shell Script für für den Bau des Latex Dokuments},language=Bash]{../../.circleci/job_document.sh}
\lstinputlisting[label={lst:pythonsh},caption={Shell Script für für den Bau des Latex Dokuments},language=Bash]{../../.circleci/job_python.sh}

\end{document}